\documentclass[fontsize=10pt,a4paper,toc=bibliography,listof=totoc, DIV=classic,BCOR=10mm]{scrbook}

\usepackage[T1]{fontenc} % T1-Fonts are better
\usepackage[utf8]{inputenc} % UTF8, bitte was sonst
%\usepackage{lmodern} % Eine etwas angenehmere Font
\usepackage{yfonts}
%\usepackage{ucs} % Extended UTF-8 input support for LaTeX
%\usepackage{cite} % Improved Citation-Handling
\usepackage{amsmath} % American Math Society Packete
\usepackage{amsfonts} % American Math Society Packete
\usepackage{amssymb} % American Math Society Packete
\usepackage{graphicx} % Erweiterter Support für Graphiken
\usepackage[ngerman]{babel} % Übersetze einige Literate auf Deutsch
\usepackage{enumerate} % Erweiterte Enumerate-Umgebungen
\usepackage{wrapfig} % Erlaubt es Text, Graphiken zu umfließen
\usepackage{caption} % % American Math Society Packete
\usepackage{listings}
\usepackage{fancyhdr}
\usepackage{xparse}
\usepackage{subcaption} % Support for sub-captions (and sub-figures)
\usepackage{float} % Verbesserte floating-objects
\usepackage{framed} % Umrahmte Abschnitte
\usepackage[framed,hyperref,amsmath,thmmarks]{ntheorem}
\usepackage{tikz} % Für LaTeX Graphiken
\usetikzlibrary{decorations.markings,decorations.pathreplacing,shapes.geometric,automata,positioning,shadows} % Ein paar nützliche TikZ-Pakete
\usepackage{ stmaryrd } % St Mary Road symbols for theoretical computer science
\usepackage[hang]{footmisc}   % Für den Einzug bei Fußnoten
\setlength{\footnotemargin}{0pt} % Setze den Einzug für die Fußnoten
\usepackage{esint}
\usepackage{chemfig}

\usepackage{makeidx} % Index für Schlagwörter
\makeindex


%\usepackage[autostyle]{csquotes} % Erweitere Anführungszeichen

\makeindex


\usepackage[unicode,pdfmenubar,linktoc=all,hidelinks,bookmarks]{hyperref} % PDF-Verklinkungen ermöglichen
\usepackage[backref=true,backend=bibtex, style=alphabetic]{biblatex}




% Dictum schöner aussehen lassen
\renewcommand*{\dictumwidth}{.45\textwidth}

% Neue Absätze nicht einziehen, das zieht sonst grauenhaft aus.
\KOMAoption{parskip}{half-}

% Abkürzende Befehle:


\newcommand{\keyword}[1]{\textbf{#1}}
\newcommand{\probl}[1]{\textsc{#1}}



% Listings

\definecolor{gray}{rgb}{0.4,0.4,0.4}
\definecolor{darkred}{rgb}{0.6,0.0,0.0}
\definecolor{cyan}{rgb}{0.0,0.6,0.6}


%\baselineskip14pt\normalfont
%\renewcommand{\baselinestretch}{1}
\bibliography{literatur}

%Makros:

\pagestyle{fancy} %eigener Seitenstil
\fancyhf{} %alle Kopf- und Fußzeilenfelder bereinigen
\fancyhead[L]{} %Kopfzeile links
\fancyhead[C]{} %zentrierte Kopfzeile
\fancyhead[R]{} %Kopfzeile rechts
\renewcommand{\headrulewidth}{0pt} %obere Trennlinie
\fancyfoot[C]{} %Seitennummer
\renewcommand{\footrulewidth}{0pt} %untere Trennlinie

\fancyfoot[OR]{\Large\frakfamily\fraklines\thepage} % "O" steht für "odd", also ungerade Seiten
\fancyfoot[EL]{\Large\frakfamily\fraklines\thepage} % "E" für "even", also gerade Seiten.
\title{\Huge Das: Giftige Buch}
\author{von Kai Sauerwald}

\begin{document}
\dedication{\textswab{\Large Danke an Walter Moers, der diese wundervolle Welt erschaffen hat.\newline Auch möchte ich Juliane danke, die mich viele Jahre begleitet hat und stets eine bereichung war.}}
\pagenumbering{roman} %römische ziffern aktiviert

\clearpage\phantom{\ }
\thispagestyle{empty}
\clearpage\phantom{\ }
\thispagestyle{empty}
\clearpage
\thispagestyle{empty}
\vspace*{\fill}
\begin{center}
	 \usetikzlibrary{calc,intersections}
\usetikzlibrary{decorations.pathmorphing,patterns}
\usetikzlibrary{through}

\begin{tikzpicture}[very thick]

\def\cd{6cm}
\def\dreh{60}
\def\dist{0.5cm}

\coordinate (d1) at (.5*\cd,0);
\coordinate (d2) at ( [shift=(120:\cd)] d1);
\coordinate (d3) at ( [shift=(240:\cd)] d2);
\coordinate (center) at ($1/3*(d1) + 1/3*(d2)+1/3*(d3)$);

\newdimen\mydim
 \newcommand\gety[1]{
      \pgfextracty\mydim{\pgfpointanchor{#1}{center}}
 }
\gety{center}
\def\len{(0.5*\cd+sqrt(0.5*\cd*0.5*\cd+\mydim*\mydim))}
%\draw (center) circle ({0.5*\cd+sqrt(0.5*\cd*0.5*\cd+\mydim*\mydim)}) ;
\node (bc) at (center) [circle,minimum size= ({(\len)*2}),draw]  {};
\node (bco) at (center) [circle,minimum size= ({(\len)*2+2*\dist}),draw]  {};
%\node (bci) at (center) [circle,minimum size=({(\len)*2}-2*\dist),draw]  {};


\node(c1) at (d1) [circle through=($(d1)!0.5!(d2)$)] {};
\node(c2) at (d2) [circle through=($(d2)!0.5!(d3)$)] {};
\node(c3) at (d3) [circle through=($(d3)!0.5!(d1)$)] {};
\coordinate(i12) at (intersection 1 of c1 and d1--d2);
\coordinate(i23) at (intersection 1 of c2 and d2--d3);
\coordinate(i31) at (intersection 1 of c3 and d3--d1);

% Berechne hier die Kantenlänge
\coordinate(o1) at ([shift=(-30:{ ((0.5*\cd+sqrt(0.5*\cd*0.5*\cd+\mydim*\mydim))) })]  center);
\coordinate(o2) at ([shift=(120-30:{ ((0.5*\cd+sqrt(0.5*\cd*0.5*\cd+\mydim*\mydim))) })]  center);
\coordinate(o3) at ([shift=(240-30:{ ((0.5*\cd+sqrt(0.5*\cd*0.5*\cd+\mydim*\mydim))) })]  center);


\draw (o1) arc [start angle=-30, end angle=-240, radius=(.5*\cd)];
\draw (o2) arc [start angle=90, end angle=-120, radius=(.5*\cd)];
\draw (o3) arc [start angle=210, end angle=0, radius=(.5*\cd)];
%\draw (c1) circle (0.5*\cd);
%\draw (c2) circle (0.5*\cd);
%\draw (c3) circle (0.5*\cd);

\draw (i12) -- (center);
\draw (i23) -- (center);
\draw (i31) -- (center);

\coordinate (bci1) at ($(o1)!\dist!(center)$);
\coordinate (bci2) at ($(o2)!\dist!(center)$);
\coordinate (bci3) at ($(o3)!\dist!(center)$);



\draw (bci1) arc [start angle=-30, end angle=-240, radius=(0.5*\cd-\dist)] arc [start angle=-60, end angle=28, radius=(.5*(\cd+2*\dist))] coordinate (bcs1) edge [bend left=45] (bci1);
\draw (bci2) arc [start angle=90, end angle=-120, radius=(.5*\cd-\dist)] coordinate (test) arc [start angle=60, end angle=148, radius=(.5*(\cd+2*\dist)] coordinate (bcs2) edge [bend left=45] (bci2);
\draw (bci3) arc [start angle=210, end angle=0, radius=(.5*\cd-\dist)] arc [start angle=-180, end angle=-92, radius=(.5*(\cd+2*\dist))] coordinate (bcs3) edge [bend left=45]  (bci3);



\draw (bcs1) edge [bend right=22.5] (o2);
\draw (bcs2) edge [bend right=22.5] (o3);
\draw (bcs3) edge [bend right=22.5] (o1);



%\draw [->,very thick,color=red] (test)  arc [->,start angle=60, end angle=210, radius=(.5*(\cd+2*\dist)] node [left] {$210^\circ$};
%\draw [->,very thick,color=blue] (test) arc [->,start angle=60, end angle=180, radius=(.5*(\cd+2*\dist)] node [left] {$180^\circ$};
%\draw [->,very thick,color=green] (test) arc [->,start angle=60, end angle=150, radius=(.5*(\cd+2*\dist)] node [left] {$150^\circ$};

%\node (bci) at (center) [red, very thick, circle,minimum size=({(\len)*2}-2*\dist),draw]  {};


%\coordinate(bc) at (intersection 1 of mypath and myarc);


%\node at (bci2) {bci2};
%\node at (bcs2) {bcs2};
%\node at (d1) {$d_1$};
%\node at (d2) {$d_2$};
%\node at (d3) {$d_3$};
%\node at (i12) {$i_{1,2}$};
%\node at (i23) {$i_{2,3}$};
%\node at (i31) {$i_{3,1}$};
%\node at (o1) {$o_1$};
%\node at (o2) {$o_2$};
%\node at (o3) {$o_3$};
%\node at (center) {$s$};

%\draw (center) -- (oo2);

%\draw (oo2) arc [start angle=85, end angle=-60, radius=(.5*\cd)];

%\draw (center) -- node [pos=0.5mm] {muh} (o1) ;

%\node at (o1) {a};
%\node at (o2) {b};
%\node at (o3) {c};
%\draw (0,0) -- (d1) -- (d2) -- (d3) -- cycle;


%\arcThroughThreePoints{oc1}{ic1}{b2};

\end{tikzpicture}
\end{center}
\vspace*{\fill}


\clearpage\phantom{\ }
\thispagestyle{empty}

\frakfamily\fraklines
\maketitle
\setcounter{page}{1}
\pagenumbering{arabic} %arbische ziffern aktiviert

\newcounter{ctra}
\setcounter{ctra}{1}
\whiledo {\value{ctra} < 332}%
{%
	\stepcounter {ctra}%
	\clearpage\ 
}
\clearpage
\setcounter{ctra}{1}
\whiledo {\value{ctra} < 107}%
{%
	\stepcounter {ctra}%
	Sie wurden s:oeben vergiftet. 
	Sie wurden s:oeben vergiftet.
	Sie wurden s:oeben vergiftet. 
	Sie wurden s:oeben vergiftet. \linebreak
}


\setcounter{ctra}{1}
\whiledo {\value{ctra} < 55}%
{%
	\stepcounter {ctra}%
	\clearpage\ 
}
\clearpage\phantom{\ }
\thispagestyle{empty}
\clearpage\clearpage\phantom{\ }
\thispagestyle{empty}
\clearpage\clearpage\phantom{\ }
\thispagestyle{empty}
\clearpage\clearpage\phantom{\ }
\thispagestyle{empty}
\clearpage

\end{document}
